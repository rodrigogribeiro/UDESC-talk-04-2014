 \documentclass{beamer}
 %\documentclass[xetex, mathserif, serif]{beamer} 
    \usepackage{agda}
    \usepackage{catchfilebetweentags}
    \usetheme{Luebeck}
    \usepackage[pdftex]{hyperref}

    %\setmainfont{XITS}
    %\setmathfont{XITS Math}
    
    \title{Programa\c{c}\~ao com Tipos Dependentes em Agda}
    \author[R. Ribeiro]{Rodrigo Ribeiro}
    \institute[UFOP]{
      Departamento de Computa\c{c}\~ao e Sistemas --- DECSI\\
      Universidade de Federal de Ouro Preto
    }
    \date{\today}
    \begin{document}
       \maketitle
       \section{Motiva\c{c}\~ao}

       \begin{frame}{Motiva\c{c}\~ao --- (I)}
         \begin{itemize}
           \item Sistemas de tipos
           \begin{itemize}
             \item T\'ecnica de verifica\c{c}\~ao de programas mais utilizada.
             \item Aplica\c{c}\~oes em seguran\c{c}a e concorr\^encia.
           \end{itemize}
           \item Por\'em, como garantir a corre\c{c}\~ao de um programa?
         \end{itemize}
       \end{frame}

       \begin{frame}{Motiva\c{c}\~ao --- (II)}
         \begin{itemize}
           \item Por\'em, como garantir a corre\c{c}\~ao de um programa?
           \begin{itemize}
             \item Geralmente, dif\'icil... 
             \item Envolve demonstra\c{c}\~oes formais ou uso de t\'ecnicas como 
                   model checking.
           \end{itemize}
         \end{itemize}
       \end{frame}

       \begin{frame}{Motiva\c{c}\~ao --- (III)}
         \begin{itemize}
           \item Verifica\c{c}\~ao de programas --- Situa\c{c}\~ao Ideal:
           \begin{itemize}
             \item Verifica\c{c}\~ao autom\'atica.
             \item Idealmente, dever\'iamos ser capazes de especificar e programar em uma mesma linguagem.
           \end{itemize}
         \end{itemize}
       \end{frame}

       \begin{frame}{Motiva\c{c}\~ao --- (IV)}
         \begin{itemize}
           \item Especificar e programar em uma mesma linguagem?
           \begin{itemize}
             \item Evita problemas de consist\^encia entre diferentes linguagens. 
           \end{itemize}
           \item Por\'em, existe esta linguagem? 
         \end{itemize}
       \end{frame}  

       \begin{frame}{Motiva\c{c}\~ao --- (V)}
         \begin{itemize}
           \item A linguagem Agda
           \begin{itemize}
             \item Linguagem Funcional!
             \item Sistema de tipos expressivo, capaz de representar especifica\c{c}\~oes.
           \end{itemize}
           \item Se tipos em Agda representam especifica\c{c}\~oes, ent\~ao...
           \begin{itemize}
             \item O processo de verifica\c{c}\~ao \'e feito pelo pr\'oprio compilador de Agda!
           \end{itemize}
         \end{itemize}
       \end{frame}

       \section{A Linguagem Agda}

       \begin{frame}{A Linguagem Agda --- (I)}
         \begin{itemize}
           \item Em Agda, n\~ao existem tipos ``built-in''. Tudo \'e definido na pr\'opria linguagem.
           \item Tipos em Agda s\~ao uma generaliza\c{c}\~ao de tipos de dados alg\'ebricos encontrados em Haskell e ML.
         \end{itemize}
       \end{frame}

       \begin{frame}{A Linguagem Agda --- (II)}
         \begin{itemize}
           \item Exemplo:

           \ExecuteMetaData[latex/Code.tex]{nat}

           \item Agda permite o uso de caracteres Unicode!
         \end{itemize}
       \end{frame}

       \begin{frame}{A Linguagem Agda --- (III)}
         \begin{itemize}
           \item Exemplo:

           \ExecuteMetaData[latex/Code.tex]{nat}

           \item A anota\c{c}\~ao \texttt{Set} especifica que $\mathbb{N}$ \'e um tipo.
           \begin{itemize}
             \item O termo \texttt{Set} possui tipo $\texttt{Set}_1$.
             \item Em Agda, temos uma hierarquia de tipos tal que: $\texttt{Set} : \texttt{Set}_1 : \texttt{Set}_2 ...$
           \end{itemize}
         \end{itemize}
       \end{frame}

       \begin{frame}{A Linguagem Agda --- (IV)}
         \begin{itemize}
           \item Exemplo:

           \ExecuteMetaData[latex/Code.tex]{nat}

           \item O tipo $\mathbb{N}$ possui dois construtores:
           \begin{itemize}
             \item \texttt{zero} que representa a constante $0$
             \item \texttt{suc} que representa a fun\c{c}\~ao de sucessor.
           \end{itemize}
         \end{itemize}
       \end{frame}       

       \begin{frame}{A Linguagem Agda --- (V)}
         \begin{itemize}
           \item Usando $\mathbb{N}$ podemos representar os n\'umeros naturais como:

           \begin{table}
             \begin{tabular}{lcl}
               zero     & $\equiv$ & 0\\
               suc zero & $\equiv$ & 1\\
               suc (suc zero) & $\equiv$ & 2 \\
               ...            & ... &    ...
             \end{tabular}
           \end{table}
         \end{itemize}
       \end{frame}

       \begin{frame}{A Linguagem Agda --- (VI)}
         \begin{itemize}
           \item Adi\c{c}\~ao em Agda:
         \end{itemize}
         \ExecuteMetaData[latex/Code.tex]{plus}
       \end{frame}

       \begin{frame}{A Linguagem Agda --- (VII)}
         \begin{itemize}
           \item Listas
         \end{itemize}
         \ExecuteMetaData[latex/Code.tex]{listdef}
       \end{frame}

       \begin{frame}{A Linguagem Agda --- (VIII)}
         \begin{itemize}
           \item Fun\c{c}\~oes sobre listas: tamanho
         \end{itemize}
         \ExecuteMetaData[latex/Code.tex]{length}
       \end{frame}
       
       \begin{frame}{A Linguagem Agda --- (IX)}
         \begin{itemize}
           \item Fun\c{c}\~oes sobre listas: concatena\c{c}\~ao
         \end{itemize}
         \ExecuteMetaData[latex/Code.tex]{concat}         
       \end{frame}

       \begin{frame}{A Linguagem Agda --- (X)}
         \begin{itemize}
           \item Podemos conjecturar que:
           \[
           \forall\,xs\,ys. length(xs ++ ys) = length\: xs + length\: ys
           \]
           \item Isto \'e, o tamanho da concatena\c{c}\~ao de duas listas \'e a soma de seus tamanhos.
         \end{itemize}
       \end{frame}
       
       \begin{frame}{A Linguagem Agda --- (XI)}
         \begin{itemize}
           \item $\forall\,xs\,ys. length(xs ++ ys) = length\: xs + length\: ys$
           \begin{itemize}
             \item Teorema facilmente provado por indu\c{c}\~ao sobre a estrutura de $xs$.
             \item Apesar de poss\'ivel provar esse fato em Agda, vamos usar o sistema de tipos de Agda
                   para garantir que a concatena\c{c}\~ao possua essa propriedade.
           \end{itemize}
           \item Para isso, vamos utilizar \textbf{tipos dependentes}.
         \end{itemize}
       \end{frame}

       \section{Tipos Dependentes}

       \begin{frame}{Tipos Dependentes --- (I)}
         \begin{itemize}
           \item Dizemos que um certo tipo \'e dependente se este depende de um valor.
           \item Exemplo --- Listas indexadas por seu tamanho:
         \end{itemize}
         \ExecuteMetaData[latex/Code.tex]{vector}
       \end{frame}

       \begin{frame}{Tipos Dependentes --- (II)}
         \begin{itemize}
           \item O tipo \texttt{Vec A n} \'e uma \textbf{fam\'ilia de tipos indexada} por n\'umeros naturais.
           \begin{itemize}
             \item Isto \'e, para cada valor \texttt{n : $\mathbb{N}$}, temos um tipo \texttt{Vec A n}
             \item Note que o tipo \texttt{Vec A n} especifica o n\'umero de elementos presentes em uma lista!
           \end{itemize}
         \end{itemize}
       \end{frame}

       \begin{frame}{Tipos Dependentes --- (III)}
         \begin{itemize}
           \item Mas, qual a vantagem disso?
           \begin{itemize}
             \item Tipos mais precisos, permitem programas corretos!
           \end{itemize}
         \end{itemize}         
       \end{frame}

       \begin{frame}{Tipos Dependentes --- (IV)}
         \begin{itemize}
           \item Exemplo: obtendo a cabe\c{c}a de uma lista.
           \begin{itemize}
             \item Problema: E se a lista for vazia, o que devemos retornar?
           \end{itemize}
         \end{itemize}
       \end{frame}

       \begin{frame}{Tipos Dependentes --- (V)}
         \begin{itemize}
           \item Primeiro, usando listas...
           \item O tipo \texttt{Maybe A} indica a possibilidade de retorno de um valor.
         \end{itemize}
         \ExecuteMetaData[latex/Code.tex]{maybe}
       \end{frame}

       \begin{frame}{Tipos Dependentes --- (VI)}
         \begin{itemize}
           \item Usando o tipo \texttt{Maybe A}:
         \end{itemize}
         \ExecuteMetaData[latex/Code.tex]{listhead}
         \begin{itemize}
           \item Problema desta solu\c{c}\~ao: Uso do tipo \texttt{Maybe A}.
         \end{itemize}
       \end{frame}

       \begin{frame}{Tipos Dependentes --- (VII)}
         \begin{itemize}
           \item Usando o tipo \texttt{Vec A n}, podemos expressar o tipo de uma lista \textbf{n\~ao vazia} por
                 \texttt{Vec A (suc n)}. 
           \item O compilador garante que jamais aplicaremos a fun\c{c}\~ao \texttt{head} a uma lista vazia!
         \end{itemize} 
         \ExecuteMetaData[latex/Code.tex]{vectorhead}
       \end{frame}

       \begin{frame}{Tipos Dependentes --- (VIII)}
         \begin{itemize}
           \item Voltando a fun\c{c}\~ao de concatena\c{c}\~ao...
         \end{itemize}
         \ExecuteMetaData[latex/Code.tex]{vectorconcat}
         \begin{itemize}
           \item Agora, a propriedade sobre o tamanho da concaten\c{c}\~ao de duas listas \'e verificada automaticamente pelo compilador.
           \item C\'odigo id\^entico a concatena\c{c}\~ao de listas.
         \end{itemize}
       \end{frame}
       
       \begin{frame}{Tipos Dependentes --- (IX)}
         \begin{itemize}
           \item Mais um exemplo: obtendo o $n$-\'esimo elemento de uma lista.
         \end{itemize}
         \ExecuteMetaData[latex/Code.tex]{lookupweak}
       \end{frame}

       \begin{frame}{Tipos Dependentes --- (X)}
         \begin{itemize}
           \item Como desenvolver uma fun\c{c}\~ao correta por constru\c{c}\~ao para esta tarefa?
           \begin{itemize}
             \item Primeiro, o que quer dizer ``ser $n$-\'esimo elemento'' de uma lista?
           \end{itemize}
         \end{itemize}
       \end{frame}

       \begin{frame}{Tipos Dependentes --- (XI)}
         \begin{itemize}
           \item Formalizando a no\c{c}\~ao de um elemento pertencer a uma lista:
         \end{itemize}
         \ExecuteMetaData[latex/Code.tex]{indef}
       \end{frame}

       \begin{frame}{Tipos Dependentes --- (XII)}
         \begin{itemize}
           \item Recuperando o \'indice de um elemento a partir de uma prova de pertin\^encia:
         \end{itemize}
         \ExecuteMetaData[latex/Code.tex]{index}
       \end{frame}

       \begin{frame}{Tipos Dependentes --- (XIII)}
         \begin{itemize}
           \item Especificando a fun\c{c}\~ao para obter o $n$-\'esimo elemento de uma lista:
           \begin{itemize}
             \item Se $n$ for uma posi\c{c}\~ao de um elemento, ent\~ao este possuir\'a uma prova de pertin\^encia.
             \item Se $n \geq \texttt{length xs}$, ent\~ao n\~ao existe esse elemento.
           \end{itemize}
         \end{itemize}
       \end{frame}

       \begin{frame}{Tipos Dependentes --- (XIV)}
         \begin{itemize}
           \item Um tipo preciso para a fun\c{c}\~ao para obter o $n$-\'esimo elemento de uma lista:
         \end{itemize}
         \ExecuteMetaData[latex/Code.tex]{lookupData}
       \end{frame}

       \begin{frame}{Tipos Dependentes --- (XV)}
         \begin{itemize}
           \item Fun\c{c}\~ao correta por constru\c{c}\~ao:
         \end{itemize}
         \ExecuteMetaData[latex/Code.tex]{lookupCorrect}
       \end{frame}

       \begin{frame}{Tipos Dependentes --- (XVI)}
         \begin{itemize}
           \item Casamento de padr\~ao com tipos dependentes:
           \begin{itemize}
             \item Se \texttt{lookup xs n = outside m}, ent\~ao $n \geq length\: xs$
             \item Se \texttt{lookup xs n = inside p}, ent\~ao $p$ \'e uma prova de que
                   $x$ pertence a lista $xs$ e $n$ \'e o \'indice obtido a partir desta prova!
           \end{itemize}
         \end{itemize}
       \end{frame}

       \section{Verifica\c{c}\~ao de Tipos de $\lambda$-termos}

       \begin{frame}{Verificando Tipos --- (I)}
         \begin{itemize}
           \item \'Ultimo e derradeiro exemplo: Algoritmo para verificar tipos correto por constru\c{c}\~ao.
           \item O tipo do algoritmo ``explica'' o porqu\^e um termo \'e correto.
           \item $\lambda$-c\'alculo
           \begin{itemize}
             \item Linguagem funcional ``minimalista''.
           \end{itemize}
         \end{itemize}
       \end{frame}

       \begin{frame}{Verificando Tipos --- (II)}
         \begin{itemize}
           \item $\lambda$-c\'alculo: Sintaxe de tipos
         \end{itemize}
         \ExecuteMetaData[latex/Code.tex]{basedefs}
       \end{frame}

       \begin{frame}{Verificando Tipos --- (III)}
         \begin{itemize}
           \item $\lambda$-c\'alculo: Sintaxe de termos
         \end{itemize}
         \ExecuteMetaData[latex/Code.tex]{rawterms}
       \end{frame}

       \begin{frame}{Verificando Tipos --- (IV)}
         \begin{itemize}
           \item Sistema de tipos
         \end{itemize}
         \ExecuteMetaData[latex/Code.tex]{derivation}
       \end{frame}
       
       \begin{frame}{Verificando Tipos --- (V)}
         \begin{itemize}
           \item Obtendo uma express\~ao a partir de sua deriva\c{c}\~ao de tipos --- type erasure
         \end{itemize}
         \ExecuteMetaData[latex/Code.tex]{erase}
       \end{frame}
       

       \begin{frame}{Verificando Tipos --- (VI)}
         \begin{itemize}
           \item Comparando dois tipos com respeito a igualdade
         \end{itemize}
         \ExecuteMetaData[latex/Code.tex]{typeeq}
       \end{frame}

       \begin{frame}{Verificando Tipos --- (VII)}
         \begin{itemize}
           \item Decidindo a igualdade entre tipos
         \end{itemize}
         \ExecuteMetaData[latex/Code.tex]{typecmp}
       \end{frame}

       \begin{frame}{Verificando Tipos --- (VIII)}
         \begin{itemize}
           \item Especificando o comportamento do algoritmo
         \end{itemize}
         \ExecuteMetaData[latex/Code.tex]{infer}
       \end{frame}

       \begin{frame}{Verificando Tipos --- (IX)}
         \begin{itemize}
           \item O algoritmo de verifica\c{c}\~ao --- parte 1:
         \end{itemize}
         \ExecuteMetaData[latex/Code.tex]{typechecker1}
       \end{frame}

       \begin{frame}{Verificando Tipos --- (X)}
         \begin{itemize}
           \item O algoritmo de verifica\c{c}\~ao --- parte 2:
         \end{itemize}
         \ExecuteMetaData[latex/Code.tex]{typechecker2}
       \end{frame}


        \begin{frame}{Verificando Tipos --- (XI)}
         \begin{itemize}
           \item O algoritmo de verifica\c{c}\~ao --- parte 3:
         \end{itemize}
         \begin{flushleft}
         \ExecuteMetaData[latex/Code.tex]{typechecker3}
         \end{flushleft}
       \end{frame}

       \section{Conclus\~ao}
       \begin{frame}{Conclus\~ao}
         \begin{itemize}
           \item Moral da hist\'oria:
           \begin{itemize}
             \item Tipos dependentes permitem a especifica\c{c}\~ao precisa do relacionamento entre par\^ametros e
                   resultados de fun\c{c}\~oes.
             \item Verificados automaticamente pelo compilador!
           \end{itemize}
           \item Isso \'e apenas a ponta do Iceberg... N\~ao falamos sobre:
           \begin{itemize}
             \item Provas em Agda, Igualdade, Termina\c{c}\~ao...
           \end{itemize}
         \end{itemize}
       \end{frame}
       \begin{frame}{C\'odigo}
         \begin{itemize}
           \item Dispon\'ivel no github: 
           \begin{center}
             \href{http://github.com/rodrigogribeiro/UDESC-talk-04-2014}{https://github.com/rodrigogribeiro/UDESC-talk-04-2014}
           \end{center}
           \item C\'odigo Agda dos exemplos e slides.
         \end{itemize}
       \end{frame}
    \end{document}